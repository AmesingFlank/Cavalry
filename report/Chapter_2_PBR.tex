\chapter{Physically Based Rendering}

In order to generate photorealistic images, renderers must find accurate solutions to the \textit{rendering equation}. This chapter explains the meaning and intuition behind the equation, and describes how it can be solved by the path-tracing algorithm.

\section{The Rendering Equation}

The rendering equation describes the strength of light at each point in space and in each direction. Formally, for each point $p$ and each direction vector $\omega$, the quantity $L(p,\omega)$ is defined to be the density of the electromagnetic flux with respect to both area and solid angle, and it measures the strength of all light rays in the direction $\omega$ that passes through $p$. In physics, this quantity is given the name \textit{radiance}, and it is measured in power per unit solid angle per unit area ($\text{W}\cdot \text{sr}^{-1}\cdot\text{m}^{-2}$).

In rendering, radiance often appears in the form $L_o(p,\omega)$ or $L_i(p,\omega)$, which mean radiance going out from the point $p$ or entering into it, respectively. More precisely, $L_o(p,\omega)$ represents the radiance that travels from $p$ and outwards in the direction $\omega$, and $L_i(p,\omega)$ represents the incoming radiance that travels towards to $p$ in the direction $-\omega$. The convention $L_i(p,\omega)$ might appear slightly counter-intuitive, since $\omega$ points in the opposite direction as the propagation of energy. However, the convenience of this notation will become apparent when the ray-tracing algorithm is formulated. 


\begin{figure}[H]
\centering




\tikzset{every picture/.style={line width=0.75pt}} %set default line width to 0.75pt        

\begin{tikzpicture}[x=0.75pt,y=0.75pt,yscale=-1,xscale=1]
%uncomment if require: \path (0,300); %set diagram left start at 0, and has height of 300

%Straight Lines [id:da6428603089338014] 
\draw    (70,220) -- (251,220) ;
\draw [shift={(160.5,220)}, rotate = 0] [color={rgb, 255:red, 0; green, 0; blue, 0 }  ][fill={rgb, 255:red, 0; green, 0; blue, 0 }  ][line width=0.75]      (0, 0) circle [x radius= 3.35, y radius= 3.35]   ;
%Straight Lines [id:da0998330538982608] 
\draw    (160.5,220) -- (218.58,162.41) ;
\draw [shift={(220,161)}, rotate = 495.24] [color={rgb, 255:red, 0; green, 0; blue, 0 }  ][line width=0.75]    (10.93,-3.29) .. controls (6.95,-1.4) and (3.31,-0.3) .. (0,0) .. controls (3.31,0.3) and (6.95,1.4) .. (10.93,3.29)   ;
%Straight Lines [id:da7334390340946089] 
\draw    (325,221) -- (506,221) ;
\draw [shift={(415.5,221)}, rotate = 0] [color={rgb, 255:red, 0; green, 0; blue, 0 }  ][fill={rgb, 255:red, 0; green, 0; blue, 0 }  ][line width=0.75]      (0, 0) circle [x radius= 3.35, y radius= 3.35]   ;
%Straight Lines [id:da024849495612862427] 
\draw    (480,160) -- (416.95,219.63) ;
\draw [shift={(415.5,221)}, rotate = 316.6] [color={rgb, 255:red, 0; green, 0; blue, 0 }  ][line width=0.75]    (10.93,-3.29) .. controls (6.95,-1.4) and (3.31,-0.3) .. (0,0) .. controls (3.31,0.3) and (6.95,1.4) .. (10.93,3.29)   ;
%Image [id:dp8436986937138125] 
\draw (161,245) node  {\includegraphics[width=27pt,height=30pt]{lightbulb.png}};
%Image [id:dp5859653893638117] 
\draw (498,154) node  {\includegraphics[width=27pt,height=30pt]{lightbulb.png}};

% Text Node
\draw (181,232) node [anchor=north west][inner sep=0.75pt]   [align=left] {$\displaystyle p$};
% Text Node
\draw (224,143) node [anchor=north west][inner sep=0.75pt]   [align=left] {$\displaystyle q$};
% Text Node
\draw (436,233) node [anchor=north west][inner sep=0.75pt]   [align=left] {$\displaystyle p$};
% Text Node
\draw (519,141) node [anchor=north west][inner sep=0.75pt]   [align=left] {$\displaystyle q$};


\end{tikzpicture}


\caption{As an example, consider two points $p$ and $q$, and direction vector $\omega=\frac{q-p}{|q-p|}$. In the left diagram, the radiance sent from $p$ to $q$ is $L_o(p,\omega)$, and in the right diagram, the radiance received by $p$ from $q$ is $L_i(p,\omega)$ }
\end{figure}


In rendering, one type of radiance that are especially important are outgoing radiances $L_o(p,\omega_o)$ where $p$ is a point on the surface of geometry. These radiances determine the appearances of surfaces as seen from the camera, and accounts for how different geometries illuminate each other. 

Surfaces in the real world reflects incoming radiances. That is, for any direction $\omega_i$, the incoming radiance $L_i(p,\omega_i)$ can contribute to the outgoing radiance $L_i(p,\omega_o)$ of another direction $\omega_o$. For each surface point, this relationship is captured by the bi-directional reflectance distribution function (BRDF), written as $f_r(p,\omega_o,\omega_i)$. Intuitively, 


For most surface, the outgoing radiance is the sum of the the radiance emitted by the surface itself and the radiance caused by surface scattering. 

For geometries that do not emit light actively, the outgoing radiances $L_o(p,\omega_o)$ are often caused by the reflection, which means part of the radiances incident 

In physically based rendering, the goal to compute the amount of radiance received by a hypothetical camera placed in the scene. In other words, for each point $p$ that is visible from the camera, the rendering algorithm computes $L_o(p,\omega_o)$, where $\omega_o$ points from $p$ towards the camera. 