\begin{abstract}

One of the ultimate goals of the field of computer graphics is to synthesize images that are indistinguishable from photographs. This task, often referred to as photorealistic rendering, is known for being extremely costly, due to the need of a physically-correct simulation of the transport of light in the scene. This project explores methods to accelerate rendering by utilizing the massively parallel computing architectures of GPUs, and creates a renderer whose performance tops some of the best renderers in the academia.

This project focuses on the widely-used simulation algorithm known as path-tracing. The project studies how to maximize the performance of path-tracing on GPUs, and implements state-of-the-art algorithms for the construction and traversal of the data structures used by this algorithm. Besides the standard path-tracing algorithm, the project also implements a newly emerged variant, where reinforcement learning is used to guide the the selection of light paths. The resulting renderer can robustly handle a variety of real-world materials, complex scene geometry, and difficult lighting situations. Moreover, comprehensive benchmarking indicates that the efficiency of this GPU renderer significantly exceeds other CPU implementations, and is even noticeably faster compared to one of the most academically renowned GPU renderer.

\end{abstract}